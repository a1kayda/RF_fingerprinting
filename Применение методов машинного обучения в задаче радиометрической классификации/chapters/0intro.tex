\chapter{Введение}
\label{chap:intro}
	\paragraph{Актуальность проблемы} 
	\noindent \\
    	При столь быстром развитии технологий передачи сигналов появляется необходимость также и в развитии технологий детектирования устройств. Существуют методы, используемые для распознавания устройств на высших уровнях, но потребность в достаточном уровне защиты и точности все еще актуальна. 
	
	\paragraph{Цель работы}
	\noindent\\
	    Для обеспечения защиты информации необходимо с высокой точностью распознавать устройства.
	    
	\paragraph{Задачи} 
	\noindent 
	\begin{itemize}
	    \item Выбор эффективного признакового описания
	    \item Выбор алгоритмов классификации в режиме обучения с учителем
	    \item Разработка и реализация алгоритма онлайн-классификации
	\end{itemize}
    	То есть, разработка алгоритма, который по сигналу с передатчика сможет идентифицировать передающее устройство.
    	
	\paragraph{Объект исследования}
	\noindent \\
    	Задача радиометрической идентификации – выделение уникального набора шумов для использования их в качестве идентификатора устройства.
    	
	\paragraph{Предмет исследования} 
	\noindent \\
    	RF fingerprint – особая форма высокочастотного сигнала, которая зависит от конкретного передающего устройства. Данный сигнал представляет из себя шум, обусловленный внутренним устройством передатчика. На практике это сигнал, полученный путем анализа преамбул принятого сигнала.
    \newpage	
	\paragraph{Практическая значимость} 
	\noindent \\
    	Радиометрическая идентификация, в отличие от других методов защиты, использует для распознавания именно самый низкий, физический уровень, поэтому обеспечивает большую безопасность, чем алгоритмы распознавания, которые осуществляются на более высоких уровнях стека сетевых протоколов OSI.
        RF fingerprint – уникальный для каждого устройства набор шумов, его практически невозможно подделать, а значит он может быть использован для предотвращения различных сетевых атак, например, MITM.

	
	
