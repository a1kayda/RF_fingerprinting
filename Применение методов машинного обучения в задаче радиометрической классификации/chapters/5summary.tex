\chapter{Заключение\\}
\label{chap:summary}
    \paragraph{Цель}
    \noindent\\
    В ходе данной работы была достигнута поставленная цель -- разработан алгоритм, позволяющий в реальном времени идентифицировать устройства с точностью >90\%.
    \paragraph{Задачи}
    \noindent
    
    \noindent
    Был произведен анализ предметной области, исследованы алгоритмы понижения размерности и классификации, а также различные способы адаптации решения к работе в реальном времени.
    
    \begin{itemize}
	    \item Подобрано эффективное признаковое описание:
	    \begin{itemize}
	        \item С помощью преобразований сигнала -- последовательного вычитания друг из друга симметричных частей спектра, полученного после взятия модуля от быстрого преобразования Фурье от частей преамбулы и дальнейшего их объединения, -- была уменьшена размерность с 480 признаков до 130 с потерей точности менее, чем 0.5\%.
	        \item Смоделирован с помощью библиотеки Keras и обучен автоэнкодер, уменьшающий размерность с 130 признаков до 20 с потерей точности менее, чем 1.5\%.
	    \end{itemize}
	    \item Произведено сравнение и подбор классификаторов:
	    \begin{itemize}
	        \item Модели, основанные на решающих деревьях дали точность 0.8-0.9, но показали себя немасштабируемыми: для того, чтобы давать хороший результат, им необходимо обучиться на большом объеме данных.
	        \item CatBoost от Yandex хорошо показала себя как модель, имеющая хорошую вариативность гиперпараметров и метрик, а также, даже без оптимизации параметров, CatBoost дает точность >0.85.
	        \item Модель MLP, схожая по устройству с автоэнкодером, показала лучший результат – Accuracy = 0.92.\\
	    \end{itemize}
	    \item Разработан и реализован алгоритм онлайн-классификации:
	    \begin{itemize}
	        \item В основу алгоритма легли Autoencoder и Multilayer Perceptrone
	        \item Обучение алгоритма происходит последовательно по эпохам на рандомизированных данных
	        \item Алгоритм имеет возможность сохранения модели для дальнейшего использования
	        \item Модель не требует большого количества памяти для сохранения
	    \end{itemize}
	\end{itemize}\\ 
    